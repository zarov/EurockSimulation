\chapter{Requirement Analysis}

\section{Authors of the project}

This project's group is composed of four I2RV students: Adrien Berthet, Isaac 
Chiboub, Thibault Miclo and Camille Mougin.

\section{Project's context}

The Eurockéennes de Belfort is a large music festival happening in the beginning
of July. It lasts three days and hosts a hundred thousand visitors each year. It
takes place in the Malsaucy peninsula - just a few miles away from the city of
Belfort. \\

Attendants can enjoy world-famous bands playing on four different stages.
On-site stands are offering food, drinks, clothes and accessories. Obviously men
and women bathrooms can also be found all over the festival. The entrance ticket
includes an access to the camp site to sleep at night.

\section{Goals of the project}

The project globally consists in simulating people's actions during one day of
festival. Our goals can be listed as follows.\\

\begin{itemize}
	\item Model the site environment: stages, stands of different kind, 
	bathrooms, trees, barriers, exit/entrance.
	\item Simulate crowd movements among stands and stages during the day, 
	according to each people particular need.
	\item At a given moment, allow simulation user to drop a bomb inside the 
	festival area. This event must lead to the evacuation of the site in a fast 
	and efficient way.
\end{itemize}

\section{Environment}

The environment will be represented in 2D. The map is a simple area, filled with
fixed entities and different kind of people. A single entrance/exit is present,
with emergency exits all around the area.\\

Entities can be listed as follows:\\

\begin{itemize}
	\item[{\bf Stages,}] where the bands play, and people gather around it
	\item[{\bf Obstacles,}] trees, barriers, etc...
	\item[{\bf Toilets,}] distinction between men’s and women’s
	\item[{\bf Food stands,}] where people go eat and drink
	\item[{\bf Shopping stands,}] where people can shop souvenirs
	\item[{\bf One entrance/exit + emergency exit,}] Where you can enter or
	leave the festival
\end{itemize}

\begin{figure}[h]
	\begin{center}
		%\includegraphics[width=0.7\textwidth]{img/img.png}
	\end{center}
	\caption{Entities representation}
\end{figure}

\section{Agents and behaviors}

\subsection{Agents}

We can define 3 types of agents :\\

\begin{itemize}
	\item[{\bf Attendant,}] somebody who is here to attend concert
	\item[{\bf Vigil,}] someone who keep an eye on people. We can have regular
	vigils, and specialized vigils like medics, who help injured people
	\item[{\bf Shopkeepers,}] sell goods to attendants people
\end{itemize}

\subsection{Behaviours}
For each agent, we can specified differents behaviour :\\

\begin{itemize}
	\item[{\bf Attendants}] each person has a list of concert he wants to see,
	he will try to go to each of them. Different schedules can be made :\\
	
		\begin{itemize}
			\item Regular schedule, one concert each hour, no conflict, the
			schedule will only be interrupted by natural needs (food, toilets,
			drink)
			\item Groupie schedule, one group in particular, skip the previous
			concert to be in the first row, then regular schedule
			\item Conflicted schedule : some concerts are in conflict, so the
			person will see x\% of the first one and 100-x\% of the second one
		\end{itemize}


 	Outside of schedules, people can act differently :\\
		
		\begin{itemize}
			\item Regular people, drink a bit, eat a bit, use the restrooms, go
			to concerts from the beginning to the end
			\item Drugged people, slow motion, can slow the evacuation process
			\item Angry people, various reasons (alcohol, pogos, fights), can
			hurt others people. People hurted should be evacuated by vigils as
			soon as seen
			\item A few, randomly selected can be heavy drinkers and start to
			act totally randomly (should be stopped by vigils as soon as seen by
			one)
			\item Idiot people, take the evacuation as an opportunity to be on
			the front row
		\end{itemize}

	\item[{\bf Vigils}] As presented above, there is two types of vigils :
		\begin{itemize}
			\item Regular vigils : stay at the same place all the day, have
			a sight of seeing to detect drunk people and angry people, maybe
			rotation of given people at given posts
			\begin{itemize}
				\item[\emph{Passive}] stationed near his post ( stage
				supervisory, entry body check)
				\item[\emph{Active}] motion until in range of troublemakers
				then warning or neutralization, back to original post after
				operation success
			\end{itemize}
			\item Medics : same as vigils but for injured people
		\end{itemize}

	\item[{\bf Shopkeepers}] sells goods to people, stay behind the stand the
	all night and evacuate when needed
\end{itemize}

Apart from those specific behaviors for each type of agent, there is also
a panic behavior, which is triggered by the user (events defined
below).

\begin{itemize}
	\item Augmentation of maximal speed
	\item Reduction of perception field
	\item Following crowd direction until perception of emergency exit
\end{itemize}

\section{Tools}

The application will be written in Java, in a Maven project. For the interface,
it will be in 2D (aerian view), using Swing. To manage the multi-agent aspect of
the project, Janus 1.0 will be used.
